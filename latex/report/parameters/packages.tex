\usepackage[square,numbers]{natbib}         % Pour la bibliographie
% \usepackage[nottoc]{tocbibind}
\usepackage{url}            % Pour citer les adresses web
\usepackage[hidelinks]{hyperref}       % Pour activer les liens cliquables
\usepackage[T1]{fontenc}    % Encodage des accents
\usepackage[utf8]{inputenc} % Lui aussi
\usepackage[english]{babel} % Pour la traduction française
\usepackage{numprint}       % Histoire que les chiffres soient bien
\usepackage{eurosym}        % Permet l'utilisation du signe € : \EUR{\num{299792458.38}}
\usepackage{appendix}       % Pour avoir des appendices
\usepackage{xspace}         % Utile lors de la création d'abréviations
\usepackage[dvipsnames]{xcolor}         % Permet de gérer facilement les couleurs.
\usepackage{verbatim}       % Permet d'insérer du code LaTeX sans qu'il soit compilé
\usepackage{algpseudocode}
\usepackage{algorithm}


% Setup du headers
\usepackage{fancyhdr}       % Fancy headers (page et titre de la section en haut)

\usepackage{etoolbox}
% \def\@part[#1]#2{%
%     \ifnum \c@secnumdepth >-2\relax
%       \refstepcounter{part}%
%       \addcontentsline{toc}{part}{\thepart\hspace{1em}#1}%
%     \else
%       \addcontentsline{toc}{part}{#1}%
%     \fi
%     \markboth{}{}% <------- the patch changed this into \partmark{#1}
%     {\centering
%      \interlinepenalty \@M
%      \normalfont
%      \ifnum \c@secnumdepth >-2\relax
%       \huge\bfseries \partname\nobreakspace\thepart
%       \par
%       \vskip 20\p@
%      \fi
%      \Huge \bfseries #2\par}%
%     \@endpart}

% \makeatletter
% \patchcmd{\@part}% <cmd>
%   {\markboth{}{}}% <search>
%   {\partmark{#1}}% <replace>
%   {}{}% <success><failure>
% \makeatother
% \pagestyle{fancy}
% \fancyhf{}
% \fancyhead{}
% \renewcommand{\headrulewidth}{0.2pt}

% \newcommand{\partmark}[1]{\markboth
%   {\color[gray]{.0}\thepart. #1}{}}

% \fancyhead[L]{Diffusion restreinte}
% \fancyhead[OR]{\nouppercase\leftmark}
% \fancyfoot[C]{\thepage}
% \lhead{Diffusion restreinte}
% \rhead{\nouppercase{\leftmark}}
% \cfoot{\thepage}

% Package pour avoir la section Annexes
\usepackage{appendix}
\renewcommand{\appendixtocname}{Appendix}  % On renomme dans la table des matières
\renewcommand{\appendixpagename}{Appendix} % On renomme dans le document

% Packages pour images et graphiques
\usepackage{graphicx}   % Inclusion des graphiques
\usepackage{graphics}
\usepackage{wrapfig}    % Dessins dans le texte.
\usepackage[justification=centering]{caption}    % Permet de personnaliser les styles des légendes.
\usepackage{subcaption} % Permet d’insérer au sein d’une figure des sous-figures, chacune d’entre elles disposant d’une légende, en plus de la légende principale.

% Un package pour les dessins (utilisé pour l'environnement {code})
\usepackage{tikz}
\usetikzlibrary{shapes.geometric, arrows}
\usepackage[framemethod=TikZ]{mdframed}

% Packages pour la création de nouvelles commandes
\usepackage{xifthen}
\usepackage{xargs}

% Packages pour la typographie
\usepackage[maxlevel=3]{csquotes} % Propose des commandes pour les citations. L'option maxlevel permet de déterminer le nombre de niveaux d'imbrication maximal.
% Exemple :  \enquote{Lorsque yyy déclare \enquote{zzz} il ne déclare rien du tout}. Les citations en bloc sont également possible : \begin{quote} citation \end{quote}. Pour les citations plus longues, on préférera \begin{quotation} citation longue \end{quotation}.
\usepackage{setspace}             % Permet de modifier l'espacement entre les lignes.
\usepackage{siunitx} % Pareil que numprint, mais pour un cadre mathématiques, donc permet de faire plus de chose : http://ctan.mines-albi.fr/macros/latex/contrib/siunitx/siunitx.pdf
\sisetup{locale = FR} % Permet de mettre les conventions françaises (séparateur des milliers et virgules pour le séparateur décimal).
% Exemples :
%\num{299792458} :  Il y a des espaces.
%\num{299792,458} : On obtient une virgule comme séparateur décimal.
%\num{.34} : Même en utilisant le point, on obtient une virgule et un zéro est placé.
%\num{299792,458 +- 0.09}  : On peut obtenir le signe "plus ou moins" avec +-.
%\num{3e8} : "e" permet d’obtenir la puissance de 10, avec le signe x (3 x 10^8)

% Packages pour les mathématiques
\usepackage{amsmath}        % La base pour les maths
\usepackage{mathrsfs}       % Quelques symboles supplémentaires
\usepackage{amssymb}        % encore des symboles.
\usepackage{amsfonts}       % Des fontes, eg pour \mathbb.
\usepackage{mathtools}      % Permet d'utiliser des symboles d'égalités spéciaux
\usepackage{dsfont}         % Pour avoir les notations nécessaires aux ensembles (nécessite mathbb)
\usepackage{stmaryrd}       % Permet de faire des intervalles avec des doubles barres (intervalles d'entiers)
\usepackage{esvect}         % Donne plein de possibilité pour les flèches des vecteurs
\usepackage{systeme}        % Permet de créer facilement des systèmes (en mode mathématique ou en texte)

\usepackage{tcolorbox}
\usepackage{enumitem}
\usepackage{witharrows}
